%
%  Introduction
% ==============
%

\chapter{Introduction}
\label{Ch:Intro}

Accelerating a particle beam in a plasma is an attractive concept because the plasma itself is
capable of sustaining significantly higher accelerating fields than converntional RF structures.
Conventional RF structures suffers electrical breakdowns at very high electric fields, and these
breakdowns can over time damage the accelerator structures \cite{braun_frequency_2003}. This puts an
upper limit on the accelerating gradient of around $350$ to $400\unit{MV/m}$. The practical upper
limit is however determined by the statistical probability of a breakdown and the acceptable number
of breakdowns in a given period of time \cite{pritzkau_experimental_2002}.

% ================================================================================================ %
\section{Plasma Wakefield Acceleration}
\label{Int:PWFA}

Intro to plasma wakefield goes here

% ================================================================================================ %
\section{Proton Driven Plasma Wakefield Acceleration}
\label{Int:PDPWFA}

Further details on proton driven plasma wakefield goes here.

% ================================================================================================ %
\section{The Self-modulation Instability}
\label{Int:SMI}

Stuff about SMI goes here

% ================================================================================================ %
\section{Numerical Simulations of PWFA}
\label{Int:Sim}

Stuff about Osiris and and all that jazz.

Reference to PIC appendix.

% ================================================================================================ %

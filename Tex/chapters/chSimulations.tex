%
%  Simulations
% =============
%

\chapter{Simulations}
\label{Ch:Sim}

% ================================================================================================ %

\section{Evolution of the Proton Beam}
\label{Sim:PBeam}

Text

% ================================================================================================ %

\subsection{Studies with Pre-modulated Beam}
\label{Sim:PBPreMod}

Text

% ================================================================================================ %

\subsection{Studies with Single Drive Bunch}
\label{Sim:PBSingle}

Text

% ================================================================================================ %

\section{Beam Loading and Energy Spread}
\label{Sim:BLoad}

Text

% ================================================================================================ %

\subsection{The Linear Regime}
\label{Sim:Lin}

The ideal case from Tzoufras 2008.

% ================================================================================================ %

\subsection{The Quasi-linear Regime}
\label{Sim:QLin}

Text

% ================================================================================================ %

\section{Emittance Evolution}
\label{Sim:Emitt}

Emittance is preserved in the linear regime

% ================================================================================================ %

\subsection{Beam Matching}
\label{Sim:Match}

Text

% ================================================================================================ %

\subsection{The Quasi-linear + Non-Linear Case}
\label{Sim:QLinNonLin}

An electron beam matched to the typical AWAKE plasma density will be, as discussed in \ref{Sim:Match}, very narrow. At a typical normalised emittance of $2.0\unit{\mu m}$ the beam is $5.25\unit{\mu m}$. Even at low beam charge and at the upper limit in terms of beam length, the wakefields of such a beam will quickly reach the non-linear regime. In the base case used in the beam loading study included in Publication \ref{Pub:BL17} \cite{berglyd_olsen:2017-1} the peak density of the beam $n_{b}/n_{0} = 35$, well beyond the saturation level of the bubble that occurs when $n_{b}/n_{0} > 10$ \cite{lu:2005}.

The implication here is that there is an additional benefitial effect of loading the accelerating field with as much charge as it will allow without overloading it. The resulting non-linear wake driven by the head of the beam, which will see emittance growth due to the quasi-linear conditions of the proton wake, ensures that the rest of the beam sees a strong focusing force preventing further emittance growth. As the electron beam gains energy, its transverse size will decrease as its emittance is preserved as $\sigma_{r} = \sqrt{\emitg\beta/\gammar}$.

% ================================================================================================ %

\section{Optimising the Witness Beam}
\label{Sim:Opt}

Bringing it all together.

% ================================================================================================ %

%
%  Simulations
% =============
%

\chapter{Simulation Analysis}
\label{Ch:SimA}

Both OSIRIS and QuickPIC dump the macro particles as large arrays of six-dimensional data, providing each particle's position and momentum vector.
A lot of work has been put into writing analysis tools to perform both initial and quick, as well as detailed analyses, of the many simulations run.
The tools developed are available online, and are described in more detail in Appendix~\ref{Apx:DA}.
Following are a couple of the key calculations done on the particle arrays.
These are for the QuickPIC data arrays, which uses equally weighted macro particles.
They also apply to the OSIRIS 3 data arrays, which uses weighted macro particles, so the weights need to be considered when performing the statistical calculations.

% ================================================================================================================================ %
\section{Extracting Twiss Parameters from Particle Arrays}
\label{SimA:EnTwiss}

To study the collective motion of particles, it is useful to calculate the bunch total emittance in terms of the RMS value or standard deviation of its particles.
Equation~\ref{EQ:EmittFull} from Section~\ref{Int:BPI:EnTwiss} can be rewritten in terms of the statistical distributions of its particles such that
\begin{equation}
    \epsilon = \sqrt{\gamma\sigma_{x}^{2} + 2\alpha\sigma_{x}\sigma_{x^{\prime}} + \beta\sigma_{x^{\prime}}^{2}}, \label{EQ:Emitt}
\end{equation}
where the angle of the $i$-th particles can be taken from its momentum
\begin{equation}
    x_{i}^{\prime} = \frac{p_{i,x}}{p_{i,z}}.
\end{equation}

For a set of macro particles, the emittance can be calculated directly by taking the covariance matrix of the $x$ and $x^{\prime}$ vectors
\begin{equation}
    \mathbf{T} = \mathrm{cov}\left(\mathbf{x}, \mathbf{x}^{\prime}\right), \label{EQ:ECalc1}
\end{equation}
and then taking the square root of its determinant
\begin{equation}
    \epsilon = \sqrt{\mathrm{det}\left(\mathbf{T}\right)}. \label{EQ:ECalc2}
\end{equation}
The Twiss parameters can be extracted from the matrix $\mathbf{T}$ as well:
\begin{equation}
    \alpha = \mathrm{T}_{12}/\epsilon, \quad
    \beta  = \mathrm{T}_{11}/\epsilon, \quad
    \gamma = \mathrm{T}_{22}/\epsilon
\end{equation}

% ================================================================================================================================ %
\section{A Measure for Beam Quality}
\label{SimA:QTilde}

For the emittance study in Publication~\ref{Pub:BL17}, it was necessary to define a convenient unit for the quality of the accelerated bunch in terms of emittance evolution in regions along the bunch length.
In the quasi-linear plus non-linear regime this publication investigates, emittance growth only occurs at the head of the bunch.
However, the region of emittance growth varies when parameters such as charge and beam size changes.
In the study, we defines the quantity
\begin{equation}
    % \tilde{Q} = \frac{1}{N} \sum_{m=0}^{M} \left[\sum_{n=0}^{N} Q_{m+n}\right] \cdot \chi(\xi_{m},N),
    \tilde{Q} = \frac{1}{N} \sum_{m=0}^{M} \sum_{n=0}^{N} Q_{m+n}\,\chi(\xi_{m},N),
\end{equation}
where $M$ is the number of longitudinal grid slices of length $\Delta\xi$ which contains macro particles for the witness bunch, and with corresponding coordinate $\xi_{m}$; $N$ is the number of such slices to average over; and $\chi(\xi_{m},N)$ is the step function
\begin{equation}
    \chi(\xi_{i},N) =
    \begin{cases}
        1, & \frac{\epsilon_{i} - \epsilon_{0}}{\epsilon_{0}} \leq 5\% \\
        0, & \frac{\epsilon_{i} - \epsilon_{0}}{\epsilon_{0}} > 5\%
    \end{cases}
    \quad\mathrm{for~}\epsilon_{i}\mathrm{~over~the~interval}\quad
    [\xi_{i}, \xi_{i} + N\Delta\xi],
\end{equation}
where $\epsilon_{i}$ is the emittance as defined by Equations~\ref{EQ:ECalc1} and~\ref{EQ:ECalc2} for a set of macro particles within the interval $\xi_{i}$ to $\xi_{i} + N\Delta\xi$, and $\epsilon_{0}$ is the initial emittance defined in the simulation input file.
For the studies included in Publication~\ref{Pub:BL17},
\begin{equation}
    M = \left\lfloor \frac{10\sigma_{z}}{\Delta\xi} \right\rceil, \quad
    N = 4.
\end{equation}
The first slice coordinate for the iterator $m$ is
\begin{equation}
    \xi_{m=0} = \mu_{\mathrm{eb}} - 5\sigma_{z,\mathrm{eb}} - 0.5\Delta\xi,
\end{equation}
where $\mu_{\mathrm{eb}}$ is the longitudinal centre of the bunch.

\paragraph{Note:} This method may yield a misleading result if the Twiss parameter $\alpha$ varies too much along the length of the bunch (the rotation of the ellipse, see Figure~\ref{Fig:BPI:Twiss}).
That is, the emittance can be locally small, and qualify for the $5\%$ criterion, even if the total emittance of the region included in $\tilde{Q}$ is not.
This can easily be checked after the seemingly optimal region of the bunch is known by verifying that its total emittance does not exceed the same criterion.

% ================================================================================================================================ %
%  Full Scale Studies
% ================================================================================================================================ %
\section{Full Scale Studies}
\label{SimA:FullScale}

A total of 38 simulations of a full AWAKE proton bunch, with and without an injected electron bunch, were run.
As can be seen from Table \ref{T:SimCost}, these simulations took an average of over $11\,000$ CPU hours.
The simulation parameters are described in Section~\ref{Sim:PBeam}.

The purpose of the full scale studies were mainly to develop some familiarity with OSIRIS 3.0, but also to study the properties of the self-modulated beam in order to make a reasonable approximation of it for the pre-modulated studies (see Section~\ref{Sim:PBPreMod}).

The structure of the self-modulation was studied with both Fast Fourier Transform (FFT)~\cite{van_loan:1992} and Morlet Wavelet Analysis~\cite{goupillaud:1984,bernardino:2005}.
Both of these tools were implemented in the OsirisAnalysis package (see Section~\ref{Tools:OAAdd}).
The FFT revealed that the core modulation frequency is indeed the characteristic frequency of the plasma.
The FFT analysis uses the entire length of the beam in its analysis, while the wavelet analysis provides a two-dimensional analysis of wave number versus the longitudinal position of the beam (add reference figures).

The frequency of the pre-modulated beam was slightly adjusted such that the FFT profiles matched that of the full scale SMI simulations \cite{berglyd_olsen:2015}.

\todo[inline]{Add wavelet and FFT figures.}

% ================================================================================================================================ %
%  Beam Loading and Energy Spread
% ================================================================================================================================ %
\section{Beam Loading and Energy Spread}
\label{SimA:BLoad}

The transverse size of the beam was chosen to be $\sigma_{x,y}=105\unit{\mu m}$, see Table \ref{T:AWAKERuns}.
The longitudinal size was chosen to be $\sigma_{z}=40\unit{\mu m}$ for the studies included in Publication \ref{Pub:IPAC15}, but several lengths were tested in simulations.
$40\unit{\mu m}$ is a good compromise between having a short enough bunch to stay within the accelerating region of the accelerating wakefield, $\approx \lambda_{pe}/4$ (see Section \ref{Int:BPI:BLoad}), and a long enough bunch to contain a reasonable amount of electrons without overloading the wakefield.

% ================================================================================================================================ %
%  Emittance Evolution
% ================================================================================================================================ %
\section{Emittance Evolution}
\label{SimA:Emitt}

Emittance is preserved in the linear regime

% ================================================================================================================================ %
\subsection{The Quasi-Linear Regime}
\label{SimA:QLin}

Text

% ================================================================================================================================ %
\subsection{The Quasi-Linear + Non-Linear Case}
\label{SimA:QLinNonLin}

Quasi-linear reference \cite{rosenzweig:2010}.

An electron beam matched to the typical AWAKE plasma density will be, as discussed in \ref{Int:BPI:Match}, very narrow. At a typical normalised emittance of $2.0\unit{\mu m}$ the beam is $5.25\unit{\mu m}$. Even at low beam charge and at the upper limit in terms of beam length, the wakefields of such a beam will quickly reach the non-linear regime. In the base case used in the beam loading study included in Publication \ref{Pub:BL17} \cite{berglyd_olsen:2018} the peak density of the beam $n_b/n_0 > 35$, well beyond the saturation level of the bubble that occurs when $n_b/n_0 > 10$ \cite{lu:2005}.

The implication here is that there is an additional beneficial effect of loading the accelerating field with as much charge as it will allow without overloading it. The resulting non-linear wake driven by the head of the beam, which will see emittance growth due to the quasi-linear conditions of the proton wake, ensures that the rest of the beam sees a strong focusing force preventing further emittance growth. As the electron beam gains energy, its transverse size will decrease as its emittance is preserved as $\sigma_{r} = \sqrt{\emitN\beta}$ \cite{wille:2001}.

% ================================================================================================================================ %
\subsection{Convergence Scan}
\label{SimA:Converge}

For the large parameter scans performed for Publication~\ref{Pub:BL17}, it was necessary to verify that the results were not dependant on grid resolution.
The radial wakefields within the plasma bubble are linear, but so are the fields within one grid cell as they are interpolated on the grid.
The effect of linear focusing could thus be an artefact of resolution.
Especially in the case where the grid cells were only a factor $2.5$ smaller than the bunch $\sigma_{x,y}$, and thus the bubble radius also small.\todo{Verify and cite.}

\begin{table}[hbt]
    \centering
    \caption{Convergence results for a reference simulations for Publication~\ref{Pub:BL17}.
    The reference bunch has a charge of $250\unit{pC}$, and the emittance tolerance criterion for the $\tilde{Q}$ parameter is $5\%$ (see Section~\ref{SimA:QTilde}).}
    \label{T:Converg}
    \begin{tabularx}{132mm}{Xl d{1}l d{1}l d{1}l}
        \rowcolor{tblhead}
        \texthh{Length} & \texthh{Param.}
            & \multicolumn{2}{c}{\texthh{1024$\times$1024}}
            & \multicolumn{2}{c}{\texthh{2048$\times$2048}}
            & \multicolumn{2}{c}{\texthh{4096$\times$4096}} \\
        \hline
                         & $\tilde{Q}$ &  213.9 & $\unit{pC}$  &  206.9 & $\unit{pC}$  &  213.1 & $\unit{pC}$  \\
        $40\unit{\mu m}$ & MEAN$(E)$   & 2263   & $\unit{MeV}$ & 2233   & $\unit{MeV}$ & 2247   & $\unit{MeV}$ \\
                         & STD$(E)$    &  267.4 & $\unit{MeV}$ &  250.4 & $\unit{MeV}$ &  261.5 & $\unit{MeV}$ \\
        \hline
                         & $\tilde{Q}$ &  221.6 & $\unit{pC}$  &  222.0 & $\unit{pC}$  &  222.1 & $\unit{pC}$  \\
        $60\unit{\mu m}$ & MEAN$(E)$   & 2346   & $\unit{MeV}$ & 2336   & $\unit{MeV}$ & 2333   & $\unit{MeV}$ \\
                         & STD$(E)$    &  166.8 & $\unit{MeV}$ &  165.0 & $\unit{MeV}$ &  165.5 & $\unit{MeV}$ \\
        \hline
                         & $\tilde{Q}$ &  229.9 & $\unit{pC}$  &  226.9 & $\unit{pC}$  &  224.8 & $\unit{pC}$  \\
        $80\unit{\mu m}$ & MEAN$(E)$   & 2378   & $\unit{MeV}$ & 2379   & $\unit{MeV}$ & 2368   & $\unit{MeV}$ \\
                         & STD$(E)$    &  120.0 & $\unit{MeV}$ &  117.6 & $\unit{MeV}$ &  119.1 & $\unit{MeV}$ \\
    \end{tabularx}
\end{table}

% ================================================================================================================================ %
\section{Optimising the Witness Beam}
\label{SimA:Opt}

Bringing it all together.

% ================================================================================================================================ %
\section{Summary of Simulation Studies}
\label{SimA:Summary}

\begin{table}[hbt]
    \centering
    \caption{Overview of total simulation cost. $97\%$ of the simulations were run on the supercomputer \textit{Abel}, on Oct Core Intel Xeon E5-2670 CPUs. The remainder were run on older nodes with Quad Core AMD Opteron 2354 CPUs.}
    \label{T:SimCost}
    \begin{tabularx}{\textwidth}{Xlrrr}
        \rowcolor{tblhead}
        \texthh{Topic of Studies}                & \texthh{Code} & \texthh{Count} &     \texthh{CPU Time} &  \texthh{Average} \\
        \hline
        Preliminary studies (mostly testing)     & OSIRIS        &           $21$ &    $266\,599\unit{h}$ & $12\,695\unit{h}$ \\
        Full length AWAKE proton bunch studies   & OSIRIS        &           $38$ &    $440\,583\unit{h}$ & $11\,594\unit{h}$ \\
        Pre-modulated beam studies$^{1}$         & OSIRIS        &          $144$ &    $319\,093\unit{h}$ &  $2\,216\unit{h}$ \\
        3D reference studies                     & OSIRIS        &           $23$ &    $245\,974\unit{h}$ & $10\,695\unit{h}$ \\
        Single drive bunch studies$^{2}$         & OSIRIS        &          $124$ &     $47\,837\unit{h}$ &     $386\unit{h}$ \\
        Beam loading and emittance studies$^{3}$ & QuickPIC      &          $293$ &    $369\,887\unit{h}$ &  $1\,262\unit{h}$ \\
        \hline
        \rowcolor{tblfoot}
        Total                                    &               &          $657$ & $1\,479\,350\unit{h}$ &  $2\,252\unit{h}$ \\
        \multicolumn{5}{p{50mm}}{\footnotesize
            $^{1}$ Main studies for Publication \ref{Pub:IPAC15} \newline
            $^{2}$ Main studies for Publication \ref{Pub:NAPAC16} \newline
            $^{3}$ Main studies for Publication \ref{Pub:BL17} \newline
        }
    \end{tabularx}
\end{table}

% ================================================================================================================================ %

%
%  Introduction
% ==============
%

\chapter{Introduction}
\label{Ch:Intro}

Accelerating a particle beam in a plasma is an attractive concept because the plasma itself is capable of sustaining
significantly higher accelerating fields than conventional RF structures. Conventional RF structures suffers electrical
breakdowns at very high electric fields, and these breakdowns can over time damage the accelerator structures
\cite{braun:2003}. This puts an upper limit on the accelerating gradient of around $350$ to $400\unit{MV/m}$. The
practical upper limit is however determined by the statistical probability of a breakdown and the acceptable number
of breakdowns in a given period of time.\cite{pritzkau:2002}

% ==================================================================================================================== %
\section{Plasma Wakefield Acceleration}
\label{Int:PWFA}

The principles behind plasma wakefield acceleration (PWFA) were formulated in the 1980s by Pisin Chen \emph{et al.}
\cite{chen:1985} Conceptually, the technique involves a beam of particles called the drive beam, travelling through
a plasma of a given density. The beam generates strong longitudinal and transverse electric fields in its wake, of witch
a trailing beam of particles draws energy in order to accelerate -- thus we see a transfer of energy from one beam to
another through plasma as the intermediate medium.

The trailing fields have a periodic structure determined by the plasma frequency, $\omega_{pe}$, and wavelength,
$\lambda_{pe}$; given as
\begin{align}
    \lambda_{pe} = \frac{2\pi c}{\omega_{pe}}, \quad
    \omega_{pe}  = \sqrt{\frac{n_{0}e^{2}}{m_{e}\epsilon_{0}}}, \label{EQ:PWFA:L0W0}
\end{align}
where $n_{0}$ is the initial plasma electron density, $e$ is the elementary charge, $m_{e}$ is the electron mass, and
$\epsilon_{0}$ is the vacuum permittivity.

It has been shown experimentally that energy can be transferred from one or two electron drive beams to a single
electron witness beam.\cite{rosenzweig:1988, blumenfeld:2007, kallos:2008} A limitation using an electron beams with a
similar initial charge and energy for both drive and witness beam is that the witness beam will rapidly gain energy
while the drive beam loses energy. This causes the witness beam to catch up with the drive beam, and the acceleration
stops.

% ==================================================================================================================== %
\subsection{The Linear Regime}
\label{Int:PWFA:Lin}

A point like charge travelling at a speed close to the speed of light, will generate fields in its wake
\cite{van_der_meer:1985,chen:1985}:
\begin{align}
    E_{z} &=            - \frac{Q_{b}k_{pe}^{2}}{2\pi\epsilon_{0}} K_{0}(k_{pe}r)\cos(k_{pe}z - \omega_{pe}t)
                          \label{EQ:PWFA:EZ} \\
    E_{r} &= \hphantom{-} \frac{Q_{b}k_{pe}^{2}}{2\pi\epsilon_{0}} K_{1}(k_{pe}r)\sin(k_{pe}z - \omega_{pe}t),
                          \label{EQ:PWFA:ER}
\end{align}
where $Q_{b}$ is the charge of the beam, $k_{pe}$ is the plasma wave number, and $K_{0}$ and $K_{1}$ are modified
Bessel functions.

% ==================================================================================================================== %
\subsection{The Non-Linear Regime}
\label{Int:PWFA:NLin}

Text

% ==================================================================================================================== %
\section{Proton Driven Plasma Wakefield Acceleration}
\label{Int:PDPWFA}

Further details on proton driven plasma wakefield goes here.

% ==================================================================================================================== %
\section{The Self-modulation Instability}
\label{Int:SMI}

Stuff about SMI goes here

% ==================================================================================================================== %
\section{Numerical Simulations of PWFA}
\label{Int:Sim}

Stuff about Osiris and and all that jazz.

Reference to PIC appendix.

% ==================================================================================================================== %

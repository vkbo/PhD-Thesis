%
%  Summary and Conclusion
% ========================
%

\chapter{Summary and Conclusion}
\label{Ch:SnC}

The aim of this thesis project was to study the performance of an optimised electron witness bunch in a proton driven plasma wakefield accelerator section.
The study was specifically aimed at Run~2 of the AWAKE experiment, scheduled after the 2019--2020 Long Shutdown~2 of the LHC, with construction planned for the shutdown period.

In Run~1 of AWAKE, which saw first beam in December 2016, the experiment sought to probe the wakefields generated by the SPS proton bunch in a $10\unit{m}$ Rubidium plasma stage.
This was done with a wide and long electron witness bunch, capable of probing the entire pase of one plasma wavelength, and a significant fraction of the radial extent of the accelerating fields.

As of the writing of this thesis, AWAKE Run~1 enters its final few months of operation.
The results of the run have been exciting, with self-modulation confirmed and studied, and $2\unit{GeV}$ accelerated electrons seen. %cite stuff

With an optimised experiment for the second run, it looks promising for even better results.
The work presented in this thesis has attempted to add insight into what can be expected for the next stage with a tuned electron witness bunch that can probe the accelerating structure more precisely and hopefully be accelerated to high energy while retaining low energy spread and emittance.



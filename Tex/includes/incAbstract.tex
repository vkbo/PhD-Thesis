\chapter*{Abstract}
\label{Abstract}

Plasma wakefield accelerators promise to deliver orders of magnitude higher accelerating gradients than conventional accelerator technology.
Whether the technology is used for even higher energy accelerators than exist today, or more compact accelerators, the promise of high gradients has sparked a number of plasma wakefield experiments over the last few decades.

The Advanced Wakefield Experiment (AWAKE) is the first to exploit the self-modulation instability in long particle bunches in plasma in combination with a proton bunch from an existing high energy synchrotron.
The experiment is located at CERN and connected to the Super Proton Synchrotron (SPS).
The first run of AWAKE saw electrons accelerated from 19 mega-electronvolts (MeV) to 2 giga-electronvolts (GeV) in just 10 metres of ionised Rubidium vapour, achieving a gradient of nearly 200 MV/m.

A challenge facing plasma wakefield accelerator designs is the final quality of the accelerated bunch in terms of its spread in energy and its emittance.
In order to minimise both these parameters while retaining a high accelerating gradient -- goals that are to an extent in conflict -- the electron bunch needs to load the generated fields in such a manner that it is as uniform as possible over the length of the bunch.
Computer simulations are needed to pinpoint the parameters that balance these opposing goals.

Part of the work included integration of the experiment into the control system at CERN.
However, most of the work presented in this thesis seeks, through computer simulations, to inform design choices for the next run of AWAKE, scheduled to start in 2021. 

The simulations show that it is, under otherwise ideal conditions, possible to accelerate 30 to 70 pico-Coulomb (pC) of electrons in an accelerator like AWAKE up to 1.8 to 2 GeV in a 4 metre plasma stage, with an energy spread of less than 2 percent and no significant emittance growth.
Low energy spread is achieved by finely tuning the witness bunch size and density to fit the plasma parameters as well as the wakefields generated by the drive bunch.
Low emittance growth is achieved by exploiting the wake generated by the head of the witness bunch to create a stable condition for the tail of the bunch.

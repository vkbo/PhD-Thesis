\chapter*{Preface}

Plasma wakefield accelerators are very complex machines, and there are many parameters to tweak in order to accelerate a particle bunch to high energies, while retaining a quality in terms of energy spread and emittance.
What practical application such accelerators may have depends on exactly what these parameters end up being, and how they depend on each other.
It is entirely possible that plasma wakefield accelerators may not produce the beam quality needed for the frontier particle physics experiments in the future>
As Terry Pratchett once said: \textit{It is well known that a vital ingredient of success is not knowing that what you're attempting can't be done}~\cite{pratchett:1987}.
That does not mean they may not be useful in other areas, like for medical applications or for other types of research.
In addition, understanding how charged particle bunches interact with plasmas is interesting on its own, and may lead to other applications.
The strong focusing forces produced by plasmas under certain conditions can be utilised by for instance plasma lenses~\cite{su:1990}, and hollow electron channels can for instance be used for beam collimation~\cite{stancari:2014}.

Computer simulations are useful when trying to understand complex systems where many factors interact.
They can be used to find and study ideal cases, or they can be used to replicate experiments in order to better understand what is going on when you cannot measure all the parameters within the experiment itself.
The work presented in this thesis is aimed towards addressing some of the questions surrounding the design of Run~2 of the AWAKE experiment at CERN (see Chapters~\ref{Ch:Intro} and~\ref{Ch:WFA}).
While the current Run~1 addresses some of the principle properties of a proton driven plasma wakefield accelerator, like the interaction between the plasma and the bunch itself, how the wakefields evolve, and how a sample of electrons behave in such a wakefield, Run~2 aims to accelerate a narrow, short electron bunch to high energies while retaining a low energy spread and low emittance.

This thesis includes an introduction outlining some of the core concepts involved in plasma wakefield acceleration techniques in Chapter~\ref{Ch:Intro}.
The AWAKE experiment itself is covered in more detail in Chapter~\ref{Ch:WFA}.
In Chapter~\ref{Ch:DAQ} some of the additional work of integrating the AWAKE experiment with the CERN Control System is outlined.
The simulations forming the basis for the publications are described in Chapters~\ref{Ch:SimS} and~\ref{Ch:SimA}, where the approximations used are also described.
A final summary and conclusion is found in Chapter~\ref{Ch:SnC}.

The four publications are included in this thesis in an appendix titled \hyperref[A:Pub]{Publications}.
Additional appendices outlining the principles of Particle in Cell codes used in this work, and a description of the analysis code written for the simulations are also included.

\section*{Publications}

\begin{enumerate}[I]
    \item Loading of a Plasma-Wakefield Accelerator Section Driven by a Self-Modulated Proton Bunch, \textit{Proceedings of IPAC 2015} \cite{berglyd_olsen:2015}
    \item Loading of Wakefields in a Plasma Accelerator Section Driven by a Self-Modulated Proton Beam, \textit{Proceedings of NAPAC 2016} \cite{berglyd_olsen:2016}
    \item Data Acquisition and Controls Integration of the AWAKE Experiment at CERN, \textit{Proceedings of IPAC 2017} \cite{berglyd_olsen:2017}
    \item Emittance Preservation of an Electron Beam in a Loaded Quasilinear Plasma Wakefield, \textit{Physical Review Accelerators and Beams} \cite{berglyd_olsen:2018}
\end{enumerate}

\section*{Notation}

Table~\ref{T:Notes} summarises some of the key notation used in this thesis that may vary in other sources covering plasma wakefield accelerators or accelerators in general.

\begin{table}[hbt]
    \centering
    \caption{Overview of notation used in this thesis.}
    \label{T:Notes}
    \begin{tabular}{p{0.12\linewidth} p{0.80\linewidth}}
        \rowcolor{tblhead}
        \texthh{Notation}             & \texthh{Description} \\
        \hline
        $n_{0}$                       & The average initial plasma density \\
        $n_{pe}$                      & The density of plasma electrons \\
        $n_{b}$                       & The density of a general particle bunch \\
        $n_{eb}$, $n_{pb}$            & The density of an electron or a proton bunch in particular \\
        $\lambda_{pe}$, $\omega_{pe}$ & The plasma wavelength and frequency$^1$ \\
        $\sigma_{r}$                  & The width of a Gaussian bunch when it is assumed to be cylindrically symmetric \\
        $\sigma_{x}$, $\sigma_{y}$    & The transverse size of a Gaussian bunch when it may not be symmetric, or the value applies to only one plane. \\
        $\alpha$, $\beta$, $\gamma$   & The Twiss parameters, also known as the Courant-Snyder parameters$^2$ \\
        $\epsilon$, $\emitN$          & Geometric and normalised emittance, respectively$^2$ \\
        $\betar$, $\gammar$           & Relativistic factors where they may be confused with the Twiss parameters \\
        $\xi$                         & The longitudinal coordinate in the reference frame of a relativistic particle bunch$^3$ \\
        \hline
        \multicolumn{2}{p{0.92\linewidth}}{\footnotesize
            $^{1}$ See Equation~\ref{EQ:PWFA:L0W0}. \newline
            $^{2}$ See Section~\ref{Int:BPI:EnTwiss}. \newline
            $^{3}$ See Equation~\ref{EQ:Xi}. \newline
        }
    \end{tabular}
\end{table}

\vfill
\chapter*{Preface}

The work presented in this thesis is aimed towards addressing some of the questions surrounding the design of Run 2 of the AWAKE experiment at CERN (see Chapters \ref{Ch:Intro} and \ref{Ch:WFA}). It is published in the following four papers:

\begin{enumerate}[I]
    \item Loading of a Plasma-Wakefield Accelerator Section Driven by a Self-Modulated Proton Bunch, \emph{Proceedings of IPAC 2015} \cite{berglyd_olsen:2015}
    \item Loading of Wakefields in a Plasma Accelerator Section Driven by a Self-Modulated Proton Beam, \emph{Proceedings of NAPAC 2016} \cite{berglyd_olsen:2016}
    \item Data Acquisition and Controls Integration of the AWAKE Experiment at CERN, \emph{Proceedings of IPAC 2017} \cite{berglyd_olsen:2017}
    \item Emittance Preservation of an Electron Beam in a Loaded Quasi-Linear Plasma Wakefield, \emph{Physical Review Accelerators and Beams} \cite{berglyd_olsen:2017a}
\end{enumerate}

The thesis also includes an introductory chapter outlining some of the core concepts involved in plasma wakefield acceleration techniques in Chapter \ref{Ch:Intro}. The AWAKE experiment itself is outlined in more detail in Chapter \ref{Ch:WFA}. The simulations forming the basis for the publications are described in Chapter \ref{Ch:Sim}, where the approximations used are also described. In Chapter \ref{Ch:DAQ} some of the additional work of integrating the AWAKE experiment with the CERN Control System is outlined. A final summary and conclusions are found in Chapter \ref{Ch:SnC}.

\noindent The four publications are included in this thesis in an appendix titled \hyperref[A:Pub]{Publications}. Additional appendices outlining the principles of Particle in Cell codes used in this work, and the analysis code written, are also included.

\newpage
\section*{Notation}

\begin{table}[hbt]
    \centering
    \caption{Overview of notation used in this thesis.}
    \label{T:Notes}
    \begin{tabular}{p{0.10\linewidth} p{0.78\linewidth}}
        \rowcolor{tblhead}
        \texthh{Notation}           & \texthh{Description} \\
        \hline
        $n_{0}$                     & The average or initial plasma density \\
        $n_{pe}$                    & The density of plasma electrons \\
        $n_{b}$                     & The density of a general particle beam \\
        $n_{eb}$, $n_{pb}$          & The density of an electron or a proton beam in particular \\
        % \hline
        $\sigma_{r}$                & The width of a Gaussian beam when it is assumed to be round \\
        $\sigma_{x}$, $\sigma_{y}$  & The transverse size of a Gaussian beam when it may not be
                                      round, or the value applies to only one plane. \\
        % \hline
        $\alpha$, $\beta$, $\gamma$ & The Twiss parameters, also known as the Courant-Snyder
                                      parameters \cite{courant:1958}. \\
        $\betar$, $\gammar$         & Relativistic factors* \\
        \hline
    \end{tabular}
\end{table}

\paragraph{*}
These are indexed with \emph{r} in this thesis to separate them from the Twiss parameters.

\vfill